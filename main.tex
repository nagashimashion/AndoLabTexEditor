\documentclass[uplatex,a4paper,12pt]{ujreport}
\usepackage{jgraduate}
%% 索引作成
%\usepackage{makeidx}
% dvipdfmxを使用しない場合はオプションを変更すること
\usepackage[dvipdfmx]{graphicx}
% 数字付きリストでラベルを使う
\usepackage{enumerate}
% 数学記号など
\usepackage{amsmath}
\usepackage{amssymb}
\usepackage{amsthm}
% URLをいい感じにする
\usepackage{url}
\usepackage{cite}

% フォント周りのおまじない
\usepackage{lmodern}
\usepackage[T1]{fontenc}
\usepackage[deluxe,jis2004]{otf}% 和文7書体の使用
\usepackage{pxchfon}

%------------------------------
% 余白設定
%------------------------------
\usepackage[left=27mm,right=27mm,top=45mm,bottom=45mm,%
 headheight=5mm,headsep=10mm,%
 footskip=12mm%
 ]{geometry}

%------------------------------
% hyperref
%------------------------------
% PDF化したときにしおりが作成され,図表へのジャンプも可能となる.
% 以下は共通
\usepackage[dvipdfm,
 bookmarks=true,
 bookmarksnumbered=true,%
 bookmarksopen=false,
 pdfstartview={FitH},%
 bookmarkstype=toc,%
 setpagesize=false,%
 hidelinks,
]{hyperref}
\usepackage{pxjahyper}

%----------------------------------------------------------------------
% 設定
%----------------------------------------------------------------------
% 目次の深さはsubsubsectionまで
\setcounter{tocdepth}{3}

% 基準となる図の幅
\newlength\figurewidth
\setlength{\figurewidth}{0.8\textwidth}
% 縦に並べた図の間の基準となるスペース
\newlength\figuresep
\setlength{\figuresep}{0.8\floatsep}

%----------------------------------------------------------------------
% 文書基本情報
%----------------------------------------------------------------------
% タイトル
\title{Overleaf用の卒論・修論テンプレート}

% 著者
\author{安藤 崇央}
\studentid{J2200}

% 所属
% 卒業論文の場合はこちら
\university{群馬大学}
% 情報学部
\department{情報学部}
\major{情報学科}
\program{計算機科学プログラム}
% \program{データサイエンスプログラム}

% 理工学部
% \department{理工学部}
% \major{電子情報理工学科}
% \major{総合理工学科}
% \course{情報科学コース}

% 修士論文の場合はこちら
% \department{大学院理工学府}
% \major{理工学専攻}
% \program{電子情報・数理教育プログラム}

% 情報学研究科の場合はこちら
% \department{大学院情報学研究科}
% \major{情報学専攻}
% \program{情報科学プログラム}

% 年度、提出年月(月までを書く)
\academicyear{令和7年度}
\date{令和8年1月}

% % 指導教員(卒論用)
\supervisor{安藤 崇央 准教授}

% 指導教員、副指導教員(修論用)、主査
% \supervisor{安藤 崇央 准教授}{太田 直哉 教授}
% \examiner{△△ ▽▽ 教授}

%======================================================================
% テキスト開始
%======================================================================
\begin{document}
% 表紙はページ番号を出力しない
\thispagestyle{empty}
% 表紙
\maketitle

%----------------------------------------------------------------------
% 概要
%----------------------------------------------------------------------
\begin{abstract}
概要をここに書く
\end{abstract}

%----------------------------------------------------------------------
% 目次
%----------------------------------------------------------------------
\tableofcontents

%----------------------------------------------------------------------
% 本文のページ番号設定
%----------------------------------------------------------------------
\setcounter{page}{0}
\pagenumbering{arabic}


%======================================================================
% 本文ここから
%======================================================================

\chapter{はじめに}
GMail\cite{gmail}は、フリーのメールサービスで…

\[
\left( \frac{1}{2} \right)
\]

\[
    \left( (\phi \to \psi) \to \omega \right)
\]


\[
    \biggl( (\mathit{fuga}) \biggr)
\]

堀らは、XXシステム\cite{hori-icaiic2020}を考案している

XXシステム\cite{hori-icaiic2020}は2020年に考案されている

ふがふが
\[
    x, y \in \mathbb{Z}
\]

\section{背景}
ここには…をかく\cite{hori-icaiic2020}

`fugu'

``hoge''

muga

\{~\}

\subsection{hoge}

\section{関連研究}

\begin{align}
    \mathit{difference} &= \mathrm{hoge} \times \mathrm{fuga} \\
    \phi &= \psi \wedge \omega
\end{align}


\begin{itemize}
    \item これこれについて書いておく
    \item あれについて実装が必要
    \item 評価はこんなふうにするか
\end{itemize}

\begin{enumerate}
    \item 順番のあるものは\verb|enumerate|環境で書く
    \item これをやったら
    \item アレをやって
    \item 次にそれもやる
\end{enumerate}

\chapter{準備}

\chapter{これこれシステム}

\chapter{評価}
\section{もげ}
\subsection{ぬがぬが}
\section{あいう}
\subsection{えお}

\chapter{まとめ}
ほげほげ


%======================================================================
% 謝辞
%======================================================================
\acknowledgment
謝辞をここに書く

%----------------------------------------------------------------------
% 参考文献
%----------------------------------------------------------------------
\bibliographystyle{junsrt}
\bibliography{ref.bib}

%======================================================================
% 付録がある場合は、\appendix 以下に書く
%======================================================================
\appendix
\chapter{アンケート}

\chapter{利用したシミュレーションログ}

\end{document}
